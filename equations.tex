% \documentclass{article}% remove option "german" for english version
% % load definitions, i.e. some useful shortcuts and notational stuff for tensors etc.
% %=========================================================================
% DEFINITIONS
%=========================================================================

% Anmerkung: \newcommand testet im Gegensatz zu \def darauf, ob ein
% Befehl bereits vorhanden ist und gibt ggf. eine Fehlermeldung aus
% so zum Beispiel
%\def\mc  #1{\mbox{$\mathcal   #1$}}
% im Vergleich zu
%\newcommand{\mc}[1]{\ensuremath{\mathcal{#1}}}

%=========================================================================

% units
\newcommand{\unit}[1]{\operatorname{#1}}


% % % % % % % % % % % % % % % % % % % % % % % % % % % % % % % % % % % % % % % % % % % % %
% % % % % % % % % % % % % % % % % % % % % % % % % % % % % % % % % % % % % % % % % % % % % 

%%
% Mathematical Operators
%%

% non-standard tensor operators
\newcommand{\Otimes}{\,\overline{\otimes}\,}
\newcommand{\Utimes}{\,\underline{\otimes}\,}
\newcommand{\norm}[1]{\|#1\|}
\newcommand{\dev}{\operatorname{dev}}
\newcommand{\trace}{\operatorname{tr}}
\def\tran{\mathrm{t}} % transposed

% derivatives
\newcommand{\Div}[1][\te X]{\nabla_{\!#1}\cdot}
\renewcommand{\div}[1][\te x]{\nabla_{\!#1}\cdot}
\newcommand{\Grad}[1][\te X]{\nabla_{\!#1}}
\newcommand{\grad}[1][\te x]{\nabla_{\!#1}}
\newcommand{\dd}[2][]{\frac{\mathrm{d}#1}{\mathrm{d}#2}}
\newcommand{\DD}[2][]{\frac{\mathrm{D}#1}{\mathrm{D}#2}}
\newcommand{\pp}[2][]{\frac{\partial#1}{\partial#2}}

% assembly-operator
\def\Ass{\mathop{\mathchoice{\AssX\huge}{\AssX\Large}{}{}}}
% helper for \Ass
\def\AssX#1{{\setbox0=\hbox{#1\sf\textbf{A}}\lower.2\ht0\copy0}}

%%%%%%%%%%%%%%%%%%%%%%%%%%%%%%%%%%%%%%%%%%%%%%%%%%%%%%%%%%%

\newcommand{\B}{\boldsymbol}
\newcommand{\sfA}{ {\textsf{A}} }
\newcommand{\BE}{{\textbf{\textsf{E}}}}
\newcommand{\BS}{{\textbf{\textsf{S}}}}
\newcommand{\E}{{{\textsf{E}}}}
\newcommand{\I}{{\textsf{I}}}
\newcommand{\Y}{{{\textsf{Y}}}}
\newcommand{\sfC}{{{\textsf{C}}}}
\newcommand{\sfS}{{{\textsf{S}}}}
\newcommand{\sfT}{{{\textsf{T}}}}
\newcommand{\BA}{{\textbf{\textsf{A}}}}
\newcommand{\BC}{{\textbf{\textsf{C}}}}
\newcommand{\BD}{{\textbf{\textsf{D}}}}
\newcommand{\BI}{{\textbf{\textsf{I}}}}
\newcommand{\BT}{{\textbf{\textsf{T}}}}
\newcommand{\be}{\begin{equation}}
\newcommand{\ee}{\end{equation}}
\newcommand{\bea}{\begin{eqnarray}}
\newcommand{\eea}{\end{eqnarray}}
\newcommand{\vareps}[0]{\varepsilon}
\newcommand{\mr}[1]{\mathrm{#1}}
\newcommand{\Mt}{{\mr{M}_\mr{t}}}
\newcommand{\Mc}{{\mr{M}_\mr{c}}}
\newcommand{\M}{{\mr{M}}}
\newcommand{\A}{{\mr{A}}}
\newcommand{\epspl}{\vareps_\mr{pl}}
\newcommand{\epstr}{\vareps_\mr{tr}}
\newcommand{\epstrdev}{\vareps_\mr{tr,dev}}
\newcommand{\epstrvol}{\vareps_\mr{tr,vol}}
\newcommand{\epsvol}{{\vareps_\mr{vol}}}
\newcommand{\epsdev}{{\vareps_\mr{dev}}}

\newcommand{\Eal}{{\textsf E}^\alpha}
\newcommand{\Evol}{{\textsf E}_\mr{vol}}
\newcommand{\Edev}{{\textsf E}_\mr{dev}}
\newcommand{\hatEvol}{\widehat{{\textsf E}}_\mr{vol}}
\newcommand{\hatEdev}{\widehat{{\textsf E}}_\mr{dev}}
\newcommand{\sigmac}{\B\sigma_\mr{mac}}
\newcommand{\half}{\frac{1}{2}}

\newcommand{  \Isym  }{  \BI^\mr{sym}  }
\newcommand{  \Idev  }{  \BI^\mr{dev}  }
\newcommand{  \Ivol  }{  \BI^\mr{vol}  }

%%%%%%%%%%%%%%%%%%%%%%%%%%%%%%%%%%%%%%%%%%%%%%%%%%%%%%%%%%%

%%
% Variable Typesets
%%

% define tensor notation
% mathchoice detects in which "surrounding" content will be placed. \mathchoice{displaystyle}{textstyle}{subscript}{subsubscript}. textstyle is inline, e.g. "normal vector $\te n$ some more text".
\makeatletter
\newcommand{\te}[1]{\mathchoice{\mbox{\boldmath$#1$}}{\mbox{\boldmath$#1$}}{\mbox{\let\f@size\sf@size\selectfont\boldmath$#1$}}{\mbox{\let\f@size\ssf@size\selectfont\boldmath$#1$}}}   % 1st order tensor
\newcommand{\tz}[1]{\mathchoice{\mbox{\boldmath$#1$}}{\mbox{\boldmath$#1$}}{\mbox{\let\f@size\sf@size\selectfont\boldmath$#1$}}{\mbox{\let\f@size\ssf@size\selectfont\boldmath$#1$}}}   % 2nd order tensor
\newcommand{\td}[1]{\mathchoice%
{% display style
    \dimen0=\f@size pt%
    \addtolength{\dimen0}{-\ssf@size pt}%
    \mbox{\raisebox{\the\dimen0}{$\scriptscriptstyle 3$}$\mathbb #1$}
}{% text style
    \dimen0=\f@size pt%
    \addtolength{\dimen0}{-\ssf@size pt}%
    \mbox{\raisebox{\the\dimen0}{$\scriptscriptstyle 3$}$\mathbb #1$}
}{% scriptstyle
    \let\f@size\sf@size\selectfont%
    \dimen0=\f@size pt%
    \addtolength{\dimen0}{-\ssf@size pt}%
    \mbox{\raisebox{\the\dimen0}{$\scriptscriptstyle 3$}$\mathbb #1$}
}{% scriptscriptstyle
    \let\f@size\ssf@size\selectfont%
    \mbox{\raisebox{1pt}{$\scriptscriptstyle 3$}$\mathbb #1$}
}}  % 3th order tensor
\newcommand{\tv}[1]{\mathchoice{\mbox{\boldmath$\mathsf #1$}}{\mbox{\boldmath$\mathsf #1$}}{\mbox{\let\f@size\sf@size\selectfont\boldmath$\mathsf #1$}}{\mbox{\let\f@size\ssf@size\selectfont\boldmath$\mathsf #1$}}}  % 4th order tensor

% define vectors and matrices
\newcommand{\vc}[1]{\mathchoice{\mbox{\underline{\boldmath$\mathrm{#1}$}}}{\mbox{\underline{\boldmath$\mathrm{#1}$}}}{\mbox{\let\f@size\sf@size\selectfont\underline{\boldmath$\mathrm{#1}$}}}{\mbox{\let\f@size\ssf@size\selectfont\underline{\boldmath$\mathrm{#1}$}}}}
\newcommand{\mx}[1]{\mathchoice{\mbox{\underline{\boldmath$\mathrm{#1}$}}}{\mbox{\underline{\boldmath$\mathrm{#1}$}}}{\mbox{\let\f@size\sf@size\selectfont\underline{\boldmath$\mathrm{#1}$}}}{\mbox{\let\f@size\ssf@size\selectfont\underline{\boldmath$\mathrm{#1}$}}}}
\makeatother

%%%%%%%%%%%%%%%%%%%%%%%%%%%%%%%%%%%%%%%%%%%%%%%%%%%%%%%%%%%

% psfrag-abkuerzungen
\newcommand{  \psr   }[1]{  \psfrag{#1}[r][r]{   ${\scriptstyle #1 \!\!}$ }  }
\newcommand{  \pst   }[1]{  \psfrag{#1}[t][t]{   ${\scriptstyle #1 }$ }  }
\newcommand{  \psrt  }[1]{  \psfrag{#1}[rt][rt]{ ${\scriptstyle #1 \!\!\!}$ }  }%


% \RequirePackage[utf8]{inputenc}
% \RequirePackage{verbatim}
% \RequirePackage{graphics,graphicx}
% \RequirePackage{fancyhdr}
% \RequirePackage{amsmath,amssymb}
% \RequirePackage[square]{natbib}
% \RequirePackage{subfigure}
% \RequirePackage{listings}
% \let\trace\relax % trace already defined but newly defined in physics package, workaround
% \RequirePackage{physics}
% \RequirePackage{hyperref}
% \RequirePackage[crop=pdfcrop,process=auto]{pstool} % irgendwann wieder auf process=all umstellen, aber auto geht schneller
% \begin{document}

\chapter{Maths and stuff}
\thispagestyle{fancy}


\label{chap:maths}

\section{Evolution Equations}
\label{sec:evoeq}

The heat treatment process is simulated with the transformation of austenite into bainite and martensite.... 

\newpage
\begin{comment}
to describe the volume fraction of martensite during phase transformation the Kostinen-Marburger model \ref{eq:KM} is utilised. The KM model is purely temperature dependent. 
\begin{align}
	\beta_M &=  1 - e^{-k\,(M_S - T)} \label{eq:KM}\\
	\dot{\beta}_M &= -e^{-k\left[M_S-T\right]} k \, \dot{T} = \left[ 1-e^{-k\left[M_S-T\right]} - 1 \right] k \, \dot{T} = - \left[ 1-\beta_M \right] k \, \dot{T}
\end{align}
For generalizing the model for more phases the term $\left[ 1-\beta_M \right]$ can be interpreted as the remaining austenite $\beta_A$ available for transformation. This yields the following rate equation 
\begin{align}
	\dot{\beta}_M = - \beta_A \, k \, \dot{T} \quad.
\end{align}
 The martensitic transformation only starts below the martensite starting temperature $M_S$ and ends at the martensite finish temperature $M_F$. The factor k is used for calibration purposes. It festlegt how much residual austenite $\beta_A^r$ remains below $M_S$ after the finished transformation
 $ k = - \ln(1-\frac{\beta_{Ar}}{\beta_A^0})/(M_S - M_F) $. 
To incorporate these conditions and ensure transformation only occurs during cooling a Heaviside function $\Gamma(x)$ is used (Hömberg, 1996; Chen et al., 1997). \\
Multiplying the following term 
\begin{equation}
	\zeta_{A \rightarrow M} = \Gamma(-\dot{T})\,\Gamma(M_S - T)\,\Gamma(T-M_f)
\end{equation}
to the rate equation for the martensitic transformation yields it's final form
\begin{equation}
	\dot{\beta_M} = - \zeta_{A \rightarrow M} \, \beta_A\, k\,\dot{T} \quad. \label{eq:AMrate}
\end{equation}

In contrast to the martensitic transformation, the bainite transformation is diffusive, time dependent.
The equation used to describe this transformation is the JMAK law developed by Johnson, Mehl, Avrami and Kolmogorov
(Avrami, 1940; Cahn, 1956). 
The JMAK law 
\begin{align}
	\beta_B = \hat{\beta}_B^{max}\,[1-e^{-b_B\,(t^{N_B})}] \label{eq:JMAK}
\end{align}
is only valid for isothermal transformation. \\
The parameters 
\begin{align}
	N_B(T) = \frac{6.1273}{\ln(\frac{t_B^f}{t_B^s})}\,\quad \text{and} \quad b_B(T) = \frac{0.01005}{(t^s_B)^{N_B(T)}}
\end{align}
are derived using the starting and finishing time $t_B^s$ and $t_B^f$ for the diffusive transformation. Those are defined at volume fraction of 1\% and 99\% of the target phase. The parameters are temperature dependent and can be determined from experimental TTT diagrams (Diss Yu 1977 oder so).
The time derivation of \ref{eq:JMAK} yields the rate equation for isothermal cooling
\begin{equation}
	\dot{\beta}_B = \hat{\beta}_B^{max}\,b_B\,N_B\,t^{(N_B-1)}\,[1-\beta_B] \label{eq:ABrate0}
\end{equation}


% Adaption of JMAK law for anisothermal
As the JMAK law in eq. \ref{eq:JMAK} is only valid for isothermal transformations it has to be adapted to depict a cooling process. The cooling curve is discretized into small isothermal steps at declining temperatures. Every step $i$ is defined by it's temperature $T_i$ and it's duration $\Delta t_i$. \\
In the following the procedure is explained using two isothermal steps of $\Delta t = 1.5s$ at the temperatures  $T1 = 600^\circ C$ and $T2 = 550^\circ C$. \\ 
The first step starts at P0 with a volume fraction of bainite 
$\beta_B = 0$ and ends at P1 after $\Delta t$ at $\beta_B^1 = 0.17$ following the transformation curve for T1.  
For the next step at T2, a fictitious point P1$^\ast$ is introduced. It is the intersection of $\beta_B = \beta_B^1$ with the transformation curve for T2. The time will be called ficitious time $t^\ast$. It is the time transformation at T2 would take to yield $\beta_B^1$.\\
Then transformation is again  following the curve for $T2$ for a $\Delta t$ of 1.5s finishing at a volume fraction $\beta_B$ of 0.78. \\


% Grafik aus phasenumwandlung_beispiele_MS.ipynb weit unten
\begin{figure}[h]
\centering
\psfragfig[width = 0.7\textwidth, trim = 0 0 0 0, clip = true]{figs/fictitiousTimeSteps}
\caption{Two isothermal cooling steps.}
\label{fig:ficTime}
\end{figure}

\begin{figure}[h]
\centering
\psfragfig[width = 0.3\textwidth, trim = 0 0 0 0, clip = true]{figs/isothermalStepsTt} 
\caption{Discretized example cooling curve.}
\label{fig:discTt}
\end{figure}

\begin{figure}[h]
\centering
\psfragfig[width = 0.9\textwidth, trim = 0 0 0 0, clip = true]{figs/isothermalSteps} 
\caption{}
\label{fig:}
\end{figure}



% include more graphics 

To incorporate the fictitious time into the rate equation the steps laid out in (Reti 2001) are done. Using the JMAK law \ref{eq:JMAK}, the fictitious time for a given volume fraction $\beta_B$ and a temperature $T$ is derived as 
\begin{equation}
	t^\ast = \left( \frac{\ln A}{b(T_{i+1})}\right)^\frac{1}{N(T_{i+1})}, \quad A = \frac{\hat{\beta}}{\hat{\beta}-\beta_i}  \quad.
\end{equation}
This relation then is inserted into the rate equation \ref{eq:ABrate} to replace the isothermal transformation time and eliminate the explicit time dependence 
\begin{equation}
	\dot{\beta} = N\,b^{\frac{1}{N}}\,(\hat{\beta}-\beta)\,\left(\ln A \right)^{1-\frac{1}{N}}\quad. \label{ABrate}
\end{equation}
To ... enable more phases the maximal volume fraction  of bainite $\hat{\beta}$ in (Oliviera 2010) is interpreted as 
\begin{equation}
	\hat{\beta} = 1 - \beta_M \quad. 
\end{equation}
This leads to a desired coupling of the rate equations of Bainite and Martensite. \\\\

\end{comment}
The initial value problem represented by the rate equations is solved by implicit time integration which leads to the evolution equation $\beta^{n+1} = \beta^n + \Delta \beta^{n+1}, \quad \text{with  } \Delta \beta^{n+1} = \Delta t \, \dot{\beta}^{n+1}$. \\
After bringing these equations into residual form 
\begin{align}
	\begin{split}
		\vc {R_{\beta}} &= \vc \beta^{n+1} - \vc \beta^n- \vc {\Delta \beta} \\
		&= \begin{bmatrix}
			R_{\beta_B}\\ 
			R_{\beta_M}
		\end{bmatrix} 		
		= \begin{bmatrix}
			\beta_B^{n+1} - \beta_B^{n} - \Delta \beta_B^{n+1} \\ 
			\beta_M^{n+1} - \beta_M^{n} - \Delta \beta_M^{n+1}
		\end{bmatrix}
	\end{split} \label{eq:betaRes}
	\\
	&\text{with  } \vc {\Delta \beta} = f(\vc \beta^{n+1})
\end{align}
the backward Newton method is used to solve the equation system \ref{eq:betaRes}. The update of the backward Newton 
\begin{equation}
	\vc \beta^{k+1} = \vc \beta^k - \frac{\vc {R_\beta}}{\vc {R_\beta \prime}}
	\label{eq:NewtonUpdate}
\end{equation}
with the iteration counter $k$ requires the derivation of the tangent 
\begin{gather}
	\vc {R_\beta \prime} = 
	\begin{bmatrix}
		\pdv{R_{\beta_B}}{\beta_B}&\pdv{R_{\beta_B}}{\beta_M}\\
		\pdv{R_{\beta_M}}{\beta_B}&\pdv{R_{\beta_M}}{\beta_M}\quad \text{with}\\ 
	\end{bmatrix}\\
	\pdv{R_{\beta_B}}{\beta_B} = 1 - \Delta t \,b^{\frac{1}{N}}\, N \left[B^{1-\frac{1}{N}} 
		- \left(1- \frac{1}{N}\right)\,B^{-\frac{1}{N}}\, \right] \\
	\pdv{R_{\beta_B}}{\beta_M} = 
		- \Delta t \,b^{\frac{1}{N}}\, N 
		\left[B^{1-\frac{1}{N}} 
		- \left(1- \frac{1}{N}\right)\,B^{-\frac{1}{N}}\,\left(1- \frac{\beta_A}{1-\beta_M}\right)
		\right]\\
	\pdv{R_{\beta_M}}{\beta_B} = -k\,\Delta T \, \zeta_{A \rightarrow M}\\
	\pdv{R_{\beta_M}}{\beta_M} = 1 - k\,\Delta T \,\zeta_{A \rightarrow M} \quad.
\end{gather}



\begin{comment}
	Volume Fraction of diffusive phases Bainite for isothermal 
\begin{gather}
	W = 1-\exp \Big(-b_B\,t^{N_B}\Big) \\
	\text{starting time: } W = 1 \% = 1-\exp({-b_B\,{t_B^s}^{N_B}}) \\
	\Leftrightarrow t_B^s = \left(\frac{\ln(99\%)}{-b}\right)^{1/N_B} = \left(\frac{0.01005}{b}\right)^{1/N_B}\\
	b_B = \frac{0.01005}{(t^s_B)^{N_B}}\\
	\frac{\ln(1-W)}{-b_B} = {t_B^s}^{N_B} \\
	\Leftrightarrow  \frac{\ln(1\%)}{\ln(99\%)} = \ln \left(\frac{t_B^s}{t_B^f}\right)\,N_B\\
	N_B = \frac{6.1273}{\ln \left(\frac{t_B^f}{t_B^s}\right)}
\end{gather}
The start and the finishing time of the diffusive transformation are defined by the following relations
\begin{align}
	W(t_B^s) = 1\% \quad \text{and} \quad W(t_B^f) = 99\%\,.
\end{align}
Inserting ... in the JMAK law yields the equations for $N_B$ and $b_B$ 


Evolution equation for diffusive trafo, see Reti and Oliviera 2010
\begin{gather}
	\beta = \hat{\beta} \, \left(1-\exp(-b \, t^N)\right)\\
	t = \left( \frac{\ln A}{b}\right)^\frac{1}{N}, \quad A = \frac{\hat{\beta}}{\hat{\beta}-\beta}\\
	\dot{\beta} = N\,b^{\frac{1}{N}}\,(\hat{\beta}-\beta)\,\left(\ln A \right)^{1-\frac{1}{N}}\\
	\hat{\beta} = 1 - \beta_M
\end{gather}
\end{comment}



% \end{document}
