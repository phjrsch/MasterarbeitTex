\chapter{Model/Theoretical Framework/Method}
\thispagestyle{fancy}
\label{chap:model}
Heat treatment problems require the solution of three problems. Heat transfer, mechanical material response and phase transformation are coupled by their physical interactions. 

\section{Phase Transformations}

Within this work two different kind of phase transformations are treated. Martensite transformation is athermal without thermal activation while the trnsformation into bainite is diffusion controlled. The bainite transformation is time dependent while martensite transformation is only driven by cooling, c.f. \cite{totten_steel_2007}. \\
%%%%%%%%%%%%%%%%%%%%%%%%%%%%%%%%%%%%%%%%%%5

For the microstructural problem two different widely accepted (sources!!!) phenomenological models are used.\\
The martensitic transformation relies on the Koistinen-Marburger model established by \cite{koistinen_general_1959}. It describes the transformtion using the following equation \\\\

-----------------------------------------------------
As dargelegt in the previous chapter athermal and diffusion-controlled phase transformations are treated in this thesis. The modelling of the athermal martensitic transformation  relies on the Koistinen-Marburger model established in \cite{koistinen_general_1959}. It describes the transformtion using the following equation 
\begin{equation}
	\beta_M =  1 - e^{-k\,(M_S - T)} \quad. \label{eq:KM}
\end{equation}
The parameters $k$ is material specific and describes the speed of the transformation. The martensite starting temperature $M_S$ depends on the material and the other factors as the local carbon content that can be influenced by other phase transformations \ref... It determines the temperature at which the transformation starts. The martensite finish temperature $M_F$ is merely a function of the other given parameters. It is defined as the temperature at which the volume fraction of martensite reaches 0.99\%.\\
Since the development of martensite is driven purely by cooling there is no time dependence. If there is no further cooling the transformation stops. \textit{This kind of transformation is called athermal because there is no thermal activation, no waiting time, opposite of isothermal. In this thesis regarded as athermal also there might be something with thermal activation going on} \\
\begin{figure}[h]%[h]
\centering
\psfragfig[ trim = 0 0 0 0, clip = true]{figs/MartDevelopment} 
\caption{Development of the martensite volume fraction under cooling.}
\label{fig:MartDevelop}
\end{figure}
Eq. \ref{eq:KM} yields the development depicted in fig. \ref{fig:MartDevelop}
The rate of the martensitic transformation can be described by 
\begin{equation}
	\dot{\beta}_M = -e^{-k\left[M_S-T\right]} \,k \, \dot{T} = \left[ 1-e^{-k\left[M_S-T\right]} - 1 \right] k \, \dot{T} = - \left[ 1-\beta_M \right] k \, \dot{T} \quad. 
\end{equation}
While this equation is valid when only considering the transformation from austenite to martensite it can be generalized for the usage with more phase transformations by interpreting the term $1-\beta_M$ as the remaining austenite $\beta_A$ available for transformation. This yields the following rate for the evolution of the martensitic phase: 
\begin{equation}
	\dot{\beta}_M = - \beta_A \, k \, \dot{T} \quad. \label{eq:rateKM}
\end{equation}

The other transformations considered in this thesis are regarded as diffusion-controlled with thermal activation. There is a time-dependence in these transformations and they do continue isothermally when the cooling is stopped. \\
The equation used to describe this transformation 

is the JMAK law developed by Johnson, Mehl, Avrami and Kolmogorov (Avrami, 1940; Cahn, 1956). 
The JMAK law 
\begin{equation}
	\beta_B = \hat{\beta}_B^{max}\,[1-e^{-b_B\,(t^{N_B})}] \label{eq:JMAK}
\end{equation}
yields the sigmoidal shape seen in fig. \ref{fig:ficTime} with a slow start and a slow finish of the transformation but a high rate in between. It is only valid for isothermal transformations. 
The parameters 
\begin{align}
	N_B(T) = \frac{6.1273}{\ln(\frac{t_B^f}{t_B^s})}\,\quad \text{and} \quad b_B(T) = \frac{0.01005}{(t^s_B)^{N_B(T)}}
\end{align}
are derived using the starting and finishing time $t_B^s$ and $t_B^f$ for the diffusive transformation. Those are defined at volume fraction of 1\% and 99\% of the target phase. The parameters are temperature dependent and can be determined from experimental TTT diagrams (Diss Yu 1977 oder so). \\
The time derivation of the JMAK law eq. \ref{eq:JMAK} yields the rate equation for isothermal cooling
\begin{equation}
	\dot{\beta}_B = \hat{\beta}_B^{max}\,b_B\,N_B\,t^{(N_B-1)}\,[1-\beta_B] \quad. \label{eq:ABrate0}
\end{equation}


% Adaption of JMAK law for anisothermal
As the JMAK law is only valid for isothermal transformations it has to be adapted to depict a cooling process. 

Something with isokinetic and additivity Scheil and Cahn. 



The cooling curve is discretized into small isothermal steps at declining temperatures. Every step $i$ is defined by it's temperature $T_i$ and it's duration $\Delta t_i$. \\
In the following the procedure is explained using two isothermal steps of $\Delta t = 1.5s$ at the temperatures  $T1 = 600^\circ C$ and $T2 = 550^\circ C$. \\ 
The first step starts at P0 with a volume fraction of bainite 
$\beta_B = 0$ and ends at P1 after $\Delta t$ at $\beta_B^1 = 0.17$ following the transformation curve for T1.  
For the next step at T2, a fictitious point P1$^\ast$ is introduced. It is the intersection of $\beta_B = \beta_B^1$ with the transformation curve for T2. The time will be called ficitious time $t^\ast$. It is the time transformation at T2 would take to yield $\beta_B^1$.\\
Then transformation is again  following the curve for $T2$ for a $\Delta t$ of 1.5s finishing at a volume fraction $\beta_B$ of 0.78. \\


% Grafik aus phasenumwandlung_beispiele_MS.ipynb weit unten
\begin{figure}[h]
\centering
\psfragfig[ trim = 0 0 0 0, clip = true]{figs/fictitiousTimeSteps} %width = 0.7\textwidth,
\caption{Two isothermal cooling steps.}
\label{fig:ficTime}
\end{figure}

\begin{figure}[h]
\centering
\psfragfig[width = 0.3\textwidth, trim = 0 0 0 0, clip = true]{figs/isothermalStepsTt} 
\caption{Discretized example cooling curve.}
\label{fig:discTt}
\end{figure}

\begin{figure}[h]
\centering
\psfragfig[ trim = 0 0 0 0, clip = true]{figs/isothermalSteps} %width = 0.9\textwidth,
\caption{}
\label{fig:}
\end{figure}



% include more graphics 

To incorporate the fictitious time into the rate equation the steps laid out in (Reti 2001) are done. Using the JMAK law \ref{eq:JMAK}, the fictitious time for a given volume fraction $\beta_B$ and a temperature $T$ the fictitious time $t^\ast $is derived as 
\begin{equation}
	t^\ast = \left( \frac{\ln A}{b(T_{i+1})}\right)^\frac{1}{N(T_{i+1})}, \quad A = \frac{\hat{\beta}}{\hat{\beta}-\beta_i}  \quad.
\end{equation}
This relation then is inserted into the rate equation \ref{eq:ABrate0} to replace the isothermal transformation time and eliminate the explicit time dependence 
\begin{equation}
	\dot{\beta} = N\,b^{\frac{1}{N}}\,(\hat{\beta}-\beta)\,\left(\ln A \right)^{1-\frac{1}{N}}\quad. \label{eq:ABrate}
\end{equation}
The generalization of the rate equation for more phases \textit{works similar} to the generalizatsion of the martensitic rate. The maximal volume fraction of bainite $\hat{\beta}$ in (Oliviera 2010) is interpreted as 
\begin{equation}
	\hat{\beta} = \beta_A = 1 - \beta_M \quad. 
\end{equation}
This leads to a desired coupling of the rate equations of Bainite and Martensite. \\\\





\subsection{Involved Phases}
Floris Osmond named Martensite after Adolf Martens in \cite{osmond_microscopic_1904}

\subsection{phenomenological models}
Kostinen Marburg

JMAK 


Bringe ich die Metallurgie in dieses Kapitel und mische sie mit den Gleichungen oder trenne ich es in zwei Kapitel? 


\section{Thermal Problem}

The thermal problem is governed by the heat equation 
