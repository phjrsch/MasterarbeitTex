%=========================================================================
% DEFINITIONS
%=========================================================================

% Anmerkung: \newcommand testet im Gegensatz zu \def darauf, ob ein
% Befehl bereits vorhanden ist und gibt ggf. eine Fehlermeldung aus
% so zum Beispiel
%\def\mc  #1{\mbox{$\mathcal   #1$}}
% im Vergleich zu
%\newcommand{\mc}[1]{\ensuremath{\mathcal{#1}}}

%=========================================================================

% units
\newcommand{\unit}[1]{\operatorname{#1}}


% % % % % % % % % % % % % % % % % % % % % % % % % % % % % % % % % % % % % % % % % % % % %
% % % % % % % % % % % % % % % % % % % % % % % % % % % % % % % % % % % % % % % % % % % % % 

%%
% Mathematical Operators
%%

% non-standard tensor operators
\newcommand{\Otimes}{\,\overline{\otimes}\,}
\newcommand{\Utimes}{\,\underline{\otimes}\,}
\newcommand{\norm}[1]{\|#1\|}
\newcommand{\dev}{\operatorname{dev}}
\newcommand{\trace}{\operatorname{tr}}
\def\tran{\mathrm{t}} % transposed

% derivatives
\newcommand{\Div}[1][\te X]{\nabla_{\!#1}\cdot}
\renewcommand{\div}[1][\te x]{\nabla_{\!#1}\cdot}
\newcommand{\Grad}[1][\te X]{\nabla_{\!#1}}
\newcommand{\grad}[1][\te x]{\nabla_{\!#1}}
\newcommand{\dd}[2][]{\frac{\mathrm{d}#1}{\mathrm{d}#2}}
\newcommand{\DD}[2][]{\frac{\mathrm{D}#1}{\mathrm{D}#2}}
\newcommand{\pp}[2][]{\frac{\partial#1}{\partial#2}}

% assembly-operator
\def\Ass{\mathop{\mathchoice{\AssX\huge}{\AssX\Large}{}{}}}
% helper for \Ass
\def\AssX#1{{\setbox0=\hbox{#1\sf\textbf{A}}\lower.2\ht0\copy0}}

%%%%%%%%%%%%%%%%%%%%%%%%%%%%%%%%%%%%%%%%%%%%%%%%%%%%%%%%%%%

\newcommand{\B}{\boldsymbol}
\newcommand{\sfA}{ {\textsf{A}} }
\newcommand{\BE}{{\textbf{\textsf{E}}}}
\newcommand{\BS}{{\textbf{\textsf{S}}}}
\newcommand{\E}{{{\textsf{E}}}}
\newcommand{\I}{{\textsf{I}}}
\newcommand{\Y}{{{\textsf{Y}}}}
\newcommand{\sfC}{{{\textsf{C}}}}
\newcommand{\sfS}{{{\textsf{S}}}}
\newcommand{\sfT}{{{\textsf{T}}}}
\newcommand{\BA}{{\textbf{\textsf{A}}}}
\newcommand{\BC}{{\textbf{\textsf{C}}}}
\newcommand{\BD}{{\textbf{\textsf{D}}}}
\newcommand{\BI}{{\textbf{\textsf{I}}}}
\newcommand{\BT}{{\textbf{\textsf{T}}}}
\newcommand{\be}{\begin{equation}}
\newcommand{\ee}{\end{equation}}
\newcommand{\bea}{\begin{eqnarray}}
\newcommand{\eea}{\end{eqnarray}}
\newcommand{\vareps}[0]{\varepsilon}
\newcommand{\mr}[1]{\mathrm{#1}}
\newcommand{\Mt}{{\mr{M}_\mr{t}}}
\newcommand{\Mc}{{\mr{M}_\mr{c}}}
\newcommand{\M}{{\mr{M}}}
\newcommand{\A}{{\mr{A}}}
\newcommand{\epspl}{\vareps_\mr{pl}}
\newcommand{\epstr}{\vareps_\mr{tr}}
\newcommand{\epstrdev}{\vareps_\mr{tr,dev}}
\newcommand{\epstrvol}{\vareps_\mr{tr,vol}}
\newcommand{\epsvol}{{\vareps_\mr{vol}}}
\newcommand{\epsdev}{{\vareps_\mr{dev}}}

\newcommand{\Eal}{{\textsf E}^\alpha}
\newcommand{\Evol}{{\textsf E}_\mr{vol}}
\newcommand{\Edev}{{\textsf E}_\mr{dev}}
\newcommand{\hatEvol}{\widehat{{\textsf E}}_\mr{vol}}
\newcommand{\hatEdev}{\widehat{{\textsf E}}_\mr{dev}}
\newcommand{\sigmac}{\B\sigma_\mr{mac}}
\newcommand{\half}{\frac{1}{2}}

\newcommand{  \Isym  }{  \BI^\mr{sym}  }
\newcommand{  \Idev  }{  \BI^\mr{dev}  }
\newcommand{  \Ivol  }{  \BI^\mr{vol}  }

%%%%%%%%%%%%%%%%%%%%%%%%%%%%%%%%%%%%%%%%%%%%%%%%%%%%%%%%%%%

%%
% Variable Typesets
%%

% define tensor notation
% mathchoice detects in which "surrounding" content will be placed. \mathchoice{displaystyle}{textstyle}{subscript}{subsubscript}. textstyle is inline, e.g. "normal vector $\te n$ some more text".
\makeatletter
\newcommand{\te}[1]{\mathchoice{\mbox{\boldmath$#1$}}{\mbox{\boldmath$#1$}}{\mbox{\let\f@size\sf@size\selectfont\boldmath$#1$}}{\mbox{\let\f@size\ssf@size\selectfont\boldmath$#1$}}}   % 1st order tensor
\newcommand{\tz}[1]{\mathchoice{\mbox{\boldmath$#1$}}{\mbox{\boldmath$#1$}}{\mbox{\let\f@size\sf@size\selectfont\boldmath$#1$}}{\mbox{\let\f@size\ssf@size\selectfont\boldmath$#1$}}}   % 2nd order tensor
\newcommand{\td}[1]{\mathchoice%
{% display style
    \dimen0=\f@size pt%
    \addtolength{\dimen0}{-\ssf@size pt}%
    \mbox{\raisebox{\the\dimen0}{$\scriptscriptstyle 3$}$\mathbb #1$}
}{% text style
    \dimen0=\f@size pt%
    \addtolength{\dimen0}{-\ssf@size pt}%
    \mbox{\raisebox{\the\dimen0}{$\scriptscriptstyle 3$}$\mathbb #1$}
}{% scriptstyle
    \let\f@size\sf@size\selectfont%
    \dimen0=\f@size pt%
    \addtolength{\dimen0}{-\ssf@size pt}%
    \mbox{\raisebox{\the\dimen0}{$\scriptscriptstyle 3$}$\mathbb #1$}
}{% scriptscriptstyle
    \let\f@size\ssf@size\selectfont%
    \mbox{\raisebox{1pt}{$\scriptscriptstyle 3$}$\mathbb #1$}
}}  % 3th order tensor
\newcommand{\tv}[1]{\mathchoice{\mbox{\boldmath$\mathsf #1$}}{\mbox{\boldmath$\mathsf #1$}}{\mbox{\let\f@size\sf@size\selectfont\boldmath$\mathsf #1$}}{\mbox{\let\f@size\ssf@size\selectfont\boldmath$\mathsf #1$}}}  % 4th order tensor

% define vectors and matrices
\newcommand{\vc}[1]{\mathchoice{\mbox{\underline{\boldmath$\mathrm{#1}$}}}{\mbox{\underline{\boldmath$\mathrm{#1}$}}}{\mbox{\let\f@size\sf@size\selectfont\underline{\boldmath$\mathrm{#1}$}}}{\mbox{\let\f@size\ssf@size\selectfont\underline{\boldmath$\mathrm{#1}$}}}}
\newcommand{\mx}[1]{\mathchoice{\mbox{\underline{\boldmath$\mathrm{#1}$}}}{\mbox{\underline{\boldmath$\mathrm{#1}$}}}{\mbox{\let\f@size\sf@size\selectfont\underline{\boldmath$\mathrm{#1}$}}}{\mbox{\let\f@size\ssf@size\selectfont\underline{\boldmath$\mathrm{#1}$}}}}
\makeatother

%%%%%%%%%%%%%%%%%%%%%%%%%%%%%%%%%%%%%%%%%%%%%%%%%%%%%%%%%%%

% psfrag-abkuerzungen
\newcommand{  \psr   }[1]{  \psfrag{#1}[r][r]{   ${\scriptstyle #1 \!\!}$ }  }
\newcommand{  \pst   }[1]{  \psfrag{#1}[t][t]{   ${\scriptstyle #1 }$ }  }
\newcommand{  \psrt  }[1]{  \psfrag{#1}[rt][rt]{ ${\scriptstyle #1 \!\!\!}$ }  }